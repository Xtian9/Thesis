\chapter{Experimental setup}
\label{chap:detector}

20-25 pages.
Intro to chapter.

\section{The Large Hadron Collider}

CERN and LHC. Franco-Swiss border. 26 km. x km underground. Largest and most 
powerful.
Collider physics. Luminosity (instantaneous/integrated) and cross 
section formulas (Marco-Andrea/Pesaresi).
Used to search for new physics and discover Higgs boson.
Need very high energy to produce high mass particles and search for new physics 
(cf xs vs com).
(Compare to ep collider and fixed target).
Proton-proton (and ion) collisions.
Bunch spacing, number of bunches, number of protons per bunch (Marco-Andrea 
table).
ATLAS, CMS, LHCb, ALICE.
Accelerator complex: hydrogen, LINAC, PS, SPS, etc (show diagram).
Centre of mass energy. Pileup (formula/estimate using inelastic xs?). 

Diagram of accelerator complex.

History of Run1, shutdown, Run 2. Amount of data delivered.

\section{The Compact Muon Solenoid}

One of two multipurpose detectors. Used to search for new physics and discover 
Higgs boson.

Hermetic coverage. Overview of subdetectors. Magnet strength.

Diagram.

Coordinate system. Pseudorapidity.

\subsection{Tracker}

How tracker works, ie electron-hole pairs (see Adam and MSci report).
High radiation environment. Little material.

Pixel tracker.

Silicon strip tracker.

TOB, TEC, etc. Their positions/extents.

Diagram of layout.

Momentum resolution, tracking efficiency, spatial resolution.

\subsection{Electromagnetic calorimeter}

Designed to detect electrons and photons. Lead tungstate crystals.

EB. EE. Preshower.

Diagram of layout.

How ECAL works (ECAL shower, bremmstrahlung and pair production - see Adam and 
APP MSci course.)

Resolution formula (a+b+c).

\subsection{Hadronic calorimeter}

Designed to detect hadrons/jets. Brass and scintillating plastic. Photodiodes. 
HF. Fibres.

HB. HE. HO. HF

Diagram of layout.

How HCAL works (hadronic showers produce scintillation light - see Adam and APP 
MSci course.)

Resolution formula.

\subsection{Muon system}

Adam: As muons are heavier than electrons, they are minimally ionising and lose 
little energy through bremsstrahlung. They therefore mostly pass through the 
ECAL and HCAL. As muons are a key component of many electroweak decays, CMS has 
a dedicated muon system interleaved with the iron return yoke surrounding the 
solenoid.
 
DT (MB), CSC (ME), RPC.

Diagram of layout.

Momentum resolution 1\%.

\subsection{Magnet}

Just one or two short paragraphs.
See Marco-Andrea, Citron, Baber.

\subsection{Trigger and data acquisition}

40 MHz.

L1 trigger.

HLT.

%\subsection{Software and computing}
% CMSSW
Computing tiers.

\subsubsection{L1 trigger upgrade and my service work?}

Not sure yet. Look at Adam, Matt, Jad.

\section{Event reconstruction}

Intro: need to put together the things observed in the detector to reconstruct 
the objects.

Object reconstruction/identification (requires revision).
Tag and probe.
vertexing/tracking, btagging, PF, antikt.
PU subtraction, isolation, cross-cleaning.
MC corrections (JECs, PU, btagging, lepton ID, etc.).

Mention objects/working points used in analysis as you go along. Maybe summary 
table like Matt. Actually maybe include this in analysis chapter (see Adam).

\subsection{Tracks and vertices}

Combinatorial track finder (CTF), Kalman filter.

Primary vertex. PU vertices. Secondary/displaced vertices (b quarks) found in 
subsequent levels of reconstruction.

Efficiencies.

Isolated tracks? (when talking about analysis objects).

\subsection{Electrons and photons}
%\subsubsection{Tag and probe/corrections}

\subsection{Muons}

\subsection{Particle flow algorithm}
Need to decide where to put this. Not clear on the connection between object 
reconstruction and particle flow.
Resources: Particle Flow paper, Particle Flow summary.

The tracks are extrapolated through the calorimeters, if they fall within the 
boundaries of one or several clusters, the clusters are associated to the 
track. The set of track and cluster(s) constitute a charged hadron and the 
building bricks are not considered anymore in the rest of the algorithm. The 
muons are identified beforehand so that their track does not give rise to a 
charged hadron. The electrons are more difficult to deal with. Indeed, due to 
the frequent Bremsstrahlung photon emission, a specific track reconstruction [3]
is needed as well as a dedicated treatment to properly attach the photon 
clusters to the electron and avoid energy double counting. Once all the tracks 
are treated, the remaining clusters result in photons in case of the 
electromagnetic calorimeter (ECAL) and neutral hadrons in the hadron 
calorimeter (HCAL). Once all the deposits

\subsection{Jets}

Briefly describe what a jet is (hadronisation, tight cone of particles [MA]).
antikt algorithm. Infrared and collinear safe.
Inputs are PF candidates. Pileup subtraction (not just here but for all 
objects).

\subsubsection{Correcting the energy of jets}

\subsubsection{b-tagging}
Identification of jets originating from bottom quarks.

\subsection{Energy sums and missing energy}

Define HT, MHT and MET. At some point (maybe near the beginning) explain why we 
use  the transverse plane (initial momentum is zero, whereas it isn't in the 
longitudinal plane).

Neutrinos do not interact in particle detectors, and therefore escape 
undetected. Their presence can be inferred by the momentum imbalance of the 
visible particles in an event. (this should be mentioned in overview of 
analysis in intro chapter).

Type-1 corrections.

\section{Data sets}

See Nick. MA has this in the analysis chapter.
We use data collected by the CMS during certain runs, plus simulated data of 
background and signal processes.

\subsection{Collected data}

35.9 fb-1. 2016. (This would be mentioned in the intro anyway).
Show lumi delivered/collected vs time.
Triggers/Primary Datasets.
SR and Muon CRs.

Maybe keep this chapter general, just have section on MC simulation, and 
mention which specific data and MC samples elsewhere.

\subsection{Monte Carlo simulation}

Description of madgraph, pythia, GEANT.

MC simulation data sets.
(Grid of masses and ctau, couplings used) - no need to specify grid, couplings 
mentioned in simplified models section.
See paper.

Describe weights and xs/lumi normalisation?
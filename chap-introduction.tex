\chapter{Introduction}
\label{chap:introduction}

%% Restart the numbering to make sure that this is definitely page #1!
\pagenumbering{arabic}

%% Note that the citations in this chapter use the journal and
%% arXiv keys: I used the SLAC-SPIRES online BibTeX retriever
%% to build my bibliography. There are also quite a few non-standard
%% macros, which come from my personal collection. You can have them
%% if you want, or I might get round to properly releasing them at
%% some point myself.
% respectively~\cite{Phys.Rev.Lett.19.1264, Phys.Rev.D2.1285,hep-ph/0410370}.

Fancy introduction to particle physics, DM, SUSY, LLPs.

Dark matter is a hypothetical form of matter which constitutes about 27\% of 
the total energy content of the universe [1]. Its nature is unknown and its 
presence has so far only been inferred indirectly through its gravitational 
effects. Large efforts by the astrophysical and particle physics communities 
are 
being made on searches for the elusive dark matter.

Mention previous results somewhere (haven't found SUSY/DM/BSM/anything new 
since Higgs boson).

First LL interpretation of search for prompt np/susy with no specific optims 
for LL. 
say why this is 
interesting/important (benchmark for future LL searches, see where sensitivity 
to LL currently lies - surprisingly already quite sensitive)

Chapter 2 will describe 
the theoretical foundations and motivations for this search. Chapter 3 will do 
bla, etc.



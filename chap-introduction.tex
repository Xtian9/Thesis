\chapter{Introduction}
\label{chap:introduction}

%% Restart the numbering to make sure that this is definitely page #1!
\pagenumbering{arabic}

%% Note that the citations in this chapter use the journal and
%% arXiv keys: I used the SLAC-SPIRES online BibTeX retriever
%% to build my bibliography. There are also quite a few non-standard
%% macros, which come from my personal collection. You can have them
%% if you want, or I might get round to properly releasing them at
%% some point myself.
% respectively~\cite{Phys.Rev.Lett.19.1264, Phys.Rev.D2.1285,hep-ph/0410370}.

%Fancy introduction to particle physics, DM, SUSY, LLPs.
%Their presence can be inferred by the momentum imbalance of the visible 
%particles in an event. (this should be mentioned in overview of analysis in 
%intro chapter).
%Mention previous results somewhere (haven't found SUSY/DM/BSM/anything new 
%since Higgs boson).

Particle physics seeks to understand the nature of the universe at a 
fundamental level. The Standard Model (SM) is a theory developed throughout 
the 20$^{\mathrm{th}}$ century that provides the current best description of 
particles and their interactions. It has been remarkably successful in 
accurately % and precisely
describing and predicting the results of many experiments. 
Despite its success, the standard model has been found to be incomplete, as it 
is not able to explain dark matter and gravity, among other things. 
One of the main goals in particle physics is therefore to develop a more 
complete theory that is able to solve the problems of the standard model.

The theory of supersymmetry (SUSY) is a popular extension of the standard model 
that attempts to do this, by introducing a new set of `supersymmetric' 
particles. 
Certain versions of supersymmetry such as `Split SUSY', as well as other 
theories beyond the standard model, additionally predict the existence of 
massive long-lived particles.

The Large Hadron Collider (LHC) at the European Organisation for Nuclear 
Research (CERN) was built with one of the primary purposes to search for 
new physics such as supersymmetry. The Compact Muon Solenoid (CMS) is one of 
the particle detectors at the LHC.

The analysis described in this thesis is a search for supersymmetry in 
collision events in the CMS detector containing jets and missing energy, a 
common signature of new physics. The results are interpreted within the context 
of split supersymmetry. This is the first-ever long-lived particle 
interpretation of a search for `prompt' new physics. 
Such a search is complementary to dedicated searches for long-lived particles 
in the cases of relatively short lifetimes (less than 10~ps). 
The results have been made public in Ref.~\cite{alphat6}.

%First LL interpretation of search for prompt np/susy with no specific optims 
%for LL. 
%say why this is 
%interesting/important (benchmark for future LL searches, see where sensitivity 
%to LL currently lies - surprisingly already quite sensitive)
%first LL interpretation of prompt analysis
%Present a novel interpretation of LL models with an inclusive search
%Broad and complementary coverage, in particular for sub-cm/sub-ns lifetimes 
%and compressed scenarios

This thesis is organised as follows. Chapter~\ref{chap:theory} provides an 
overview of the standard model, supersymmetry and exotic long-lived particles, 
as well as the current status of searches for new physics. 
Chapter~\ref{chap:detector} describes the experimental apparatus of the LHC and 
CMS, as well as the particle reconstruction techniques. 
Chapter~\ref{chap:analysis} describes the analysis of this search, including 
the event selection and the background rejection and estimation. 
Chapter~\ref{chap:results} presents the results obtained and the 
interpretations. 
Finally, the content of the thesis is summarised in 
Chapter~\ref{chap:conclusion}.

%1In this chapter, the conventional/natural units } = c = 1 have been used.
%2The electric charge is measured in units of electron charge e = 1.602  10 -19 
%C.
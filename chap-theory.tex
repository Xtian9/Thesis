\chapter{Theory and motivation}
\label{chap:theory}

%% Note that the citations in this chapter use the journal and
%% arXiv keys: I used the SLAC-SPIRES online BibTeX retriever
%% to build my bibliography. There are also quite a few non-standard
%% macros, which come from my personal collection. You can have them
%% if you want, or I might get round to properly releasing them at
%% some point myself.
% respectively~\cite{Phys.Rev.Lett.19.1264, Phys.Rev.D2.1285,hep-ph/0410370}.

%\chapterquote{Laws were made to be broken.}%
%{Christopher North, 1785--1854}%: Blackwood's Magazine May 1830

Some introductory paragraph. We will begin by looking at the SM. Then bla and 
finally bla. I wouldn’t have so much SM math - brief intro to SM (each term in 
Lagrangian) and then move on to BSM (DM, SUSY, LL, simplified models, current 
limits). Theory and motivation (15\%) (15 pages – 2 pages per bullet below)

\section{The standard model of particle physics}
\label{sec:theory-sm}
3 pages. Brief intro to SM - see Marco-Andrea two pages.
If have time/space/botheredness, mention SM Lagrangian and SU3xSU2xU1. QED and 
QCD.

The Standard Model (SM) is a quantum field theory which describes the 
fundamental particles (quarks, leptons, gauge bosons and Higgs boson) and their 
interactions (electromagnetic, weak nuclear and strong nuclear).
Fermions (half-integer spin, matter) and bosons (integer spin, force carriers).

SM particles summarised in Table. We will describe in the following subsections.
Table summarising all SM particles with their mass and charge (Marco-Andrea, 
Adam, Nick).

\subsection{Fundamental fermions}
Quarks (6) and leptons (6), generations, charges, interactions.

\subsection{Fundamental bosons}
Quanta of gauge fields. Mediators of em, weak, strong interactions. Describe 
each interaction in turn. Gravity not included in SM. Ranges. Which particles 
feel each force.

Electroweak symmetry breaking. Higgs boson (scalar, spin 0). Discovered in 2012 
(final piece of SM).

\section{Beyond the standard model}
\label{sec:theory-bsm}
1.5 pages. Problems with the standard model. Tapper slides. 
MA and AE.

The SM has been well tested experimentally and has been found to describe very 
accurately a wide range of physical phenomena. However, it does not incorporate 
the gravitational force, account for neutrino oscillations or explain the 
matter-antimatter asymmetry in the universe. There are also theoretical 
concerns about the lack of unification of forces at the Grand Unified Theory
(GUT) scale as well as the hierarchy problem, whereby higher order loop 
corrections to the mass of the Higgs boson lead to a divergence in its mass 
unless a large amount of `un-natural' fine tuning is introduced. Finally, and 
most relevant to this discussion, no particle in the SM is a viable candidate 
for dark matter. The SM is therefore thought to be an effective theory which is 
only valid at low energies and a more complete model at larger energy scales 
must exist.
Gravitational force, Planck scale

Neutrino masses

Dark matter. Makes up large proportion of universe. Lots of evidence. No 
particle in SM is DM candidate. Discussed in more detail in 
Sec.~\ref{sec:theory-dm}. Or if chapter ends up being too long summarise and 
merge that section into here (maybe make a subsection)

Higgs mass/hierarchy problem

Gauge coupling unification

Matter-antimatter asymmetry

\section{Dark matter}
\label{sec:theory-dm}
DELETE - DARK MATTER NO LONGER IN THESIS.

1-1.5 pages. More details on DM (evidence, WIMP miracle, cosmology) (borrow 
from my reports)

Evidence -- see my report.

The most favoured DM candidate is a non-baryonic, weakly interacting massive 
particle (WIMP) which is stable and electrically neutral [8]. WIMPs fall into 
the category of cold dark matter, meaning that they were non-relativistic at 
the time of freeze-out and hence lead to the large scale structures observed in 
the universe today. Their weak interaction cross section also results in the 
correct relic abundance required to explain the present dark matter content of 
the universe (the ``WIMP miracle'').

Searching for DM (report): DD, ID, collider (complimentary). 

\section{Supersymmetry}
\label{sec:theory-susy}
1.5-2 pages. SUSY theory (borrow from RA1/kostas theses and RA1 papers).
MA and AE.

Poincare group, additional symmetry between bosons and fermions.
Broken symmetry.
Solves hierarchy problem, unification, DM.

MSSM is simplest version of SUSY (minimal particles). 105 parameters.
Table of MSSM particles (MA).
R-parity. Relationship with DM – LSP/neutralino is DM candidate.
stop naturalness?

\begin{comment}
MA intro: In its simplest form, supersymmetry doubles the particle spectrum, 
with the new superpartners having a spin difference of half a unit with respect 
to their Stan dard Model counterparts, but sharing all other quantum numbers. 
At the time of writing, no supersymmetric particles have been discovered, which 
implies that the Standard Model particles and their superpartners must have 
different masses, and the symmetry must therefore be broken.
\end{comment}
\begin{comment}
Models of supersymmetry [1, 2] (SUSY) with a stable, neutral, massive, weakly 
interacting, lightest supersymmetric particle (LSP) have received considerable 
attention in recent years [3] because they simultaneously offer a solution to 
the hierarchy problem, allow unification of the fundamental interactions, and 
provide a dark matter candidate
arXiv:1411.7255
\end{comment}

A theoretically well motivated extension to the Standard Model is supersymmetry
(SUSY) [6, 7], which introduces a new spacetime symmetry between fermions and 
bosons. Essentially, for every fundamental particle there exists a 
supersymmetric partner with spin differing by 12. Fermions and bosons enter the 
mass correction calculation of the Higgs boson with opposite signs, so if the 
mass difference between SM particles and their superpartners is not too large 
(ab 1 TeV), an incomplete cancellation between the two leads to the naturally 
small Higgs boson mass. Thus, supersymmetry can provide a solution to the 
hierarchy problem of the Standard Model. Furthermore, the strong, weak and 
electromagnetic couplings are found to converge at the GUT scale, thus providing
a solution to the unification problem as well. Finally, most supersymmetric 
theories predict a particle candidate for dark matter, as explained in the next 
section.

R-parity is a quantity defined such that Standard Model and supersymmetric 
particles are assigned PR = +1 and PR = -1, respectively. If R-parity is 
conserved, the lightest supersymmetric particle (LSP) cannot decay and thus is 
stable. If it is neutral, this LSP satisfies the WIMP criteria. The LSP is 
usually assumed to be the neutralino, a superposition of the superpartners of 
the Higgs boson and gauge bosons. Other dark matter candidates include the 
axion and the lightest Kaluza-Klein particle (LKP) arising from solutions to 
the strong CP problem and theories of extra spatial dimensions, respectively.

\section{Exotic long-lived particles}
\begin{comment}
More details on LL theories (see LLP thesis, rob slides, my IC/SUSY slides 
etc.), R-hadron, displaced-X searches, GMSB? Note that LLPs exist in SM – call 
them BSMLLPs or LL DM/SUSY signatures? – no, exotic LLPs!
\end{comment}
1 page.

Many BSM theories predict long-lived particles. There are examples in the SM 
(give examples). They travel some distance before decaying. Could be within or 
outside the detector. Decay length follows an exponential distribution (give 
pdf). define ctau. Width is inverse of lifetime.

Briefly describe MSSM (NLSP), GMSB, hidden valleys, RPV SUSY.

%https://indico.cern.ch/event/671803/contributions/2788342/attachments/1568495/2473041/Displaced_Jets_Brussels_Exotica.pdf
slide 2: 
• RPV SUSY
• Stealth SUSY
• Split SUSY
• WIMP Baryogenesis
• Neutral Naturalness
• Axions
• Right Handed Neutrinos
• Many more…


DMLL useful presentation (eg slide 11)
%https://indico.cern.ch/event/671803/contributions/2782601/attachments/1568796/2473998/2017-11-31_exo_lldm.pdf


\subsection{Split supersymmetry}
LLP thesis 4.2.4 and paper.
SUSY borken near unification scale. Squarks decoupled to very 
high scale. Gluinos are long-lived. Coloured. Must 
decay via highly virtual squarks. Lifetime larger than hadronisation scale 
(ps?). Gluinos form R-hadrons (so-called because of non-trivial R-parity) (can 
be mesons, baryons or gluinoballs). Can change charge through nuclear 
interactions.


\section{Simplified models}
\label{sec:theory-simplifiedmodels}
1.5 pages. Motivation for simplified models (see 3 reasons in OB LLDM, too many 
parameters in MSSM. CMSSM not model-independent).

General SUSY signature at LHC: Pair production, strong interaction dominant at 
LHC (squarks and gluinos), decay to SM particles and LSP, large mass hence 
large MET (briefly describe MET - momentum imbalance due to LSP not being 
detected).

Only one decay considered, other particles at much higher scale.

These are the simplified models that are used in this thesis to interpret the 
results in this thesis.

3 params: two masses and lifetime.
The long-lived particle's lifetime is a parameter of the model. Consider life 
distances (lifetimes) from ctau = 1 um (tau = x ns) to 100 m (tau = y ns), plus 
prompt and stable.
% and is expressed in units of [distance]/c. For example, ctau = 1 mm means... 
%(give pdf of decay length).

%The simulation of these models is described in Sec X.

%%%\subsection{Simplified models of supersymmetry}

T1qqqq and T1qqqqLL (split SUSY): here consider (long-lived) gluino pair 
production, each decays to two (light) quarks and an LSP. As gluino lifetime 
goes to 0 recover prompt model.

Interaction of R-hadron with detector material not simulated (won't have an 
effect on our analysis/not the purpose of our analysis -- see paper). If 
necessary, describe cloud model, and percentages of gluinoball etc (and/or 
describe in simulation section).

Feynman diagram of T1qqqqLL %(a) and dDM (need to make it myself?) (b) together.

%\subsection{Simplified models of dark matter}
%Buch.
%DMF, DMSimp?, dDM vector/axial

%Run 2 DM searches interpreted using simplified ``mono-X'' models with V/A/S/P 
%s-channel mediator pair decaying to Dirac fermion DM, with 4 parameters (two 
%masses and couplings).
%(Lagrangians probably not necessary)

%Extend these models to incorporate long-lived neutral particles. Kevin/me: 
%instead of mediator decaying to stable DM, it decays intermediate chi2 that is 
%long-lived and then chi2 decays to SM particles and stable DM. As chi2 
%%%lifetime 
%goes to infinity recover stable model.

%EFT description of chi2 decay. Lambda parameter can be swapped for lifetime 
%parameter.

%In this thesis consider model with vector mediator, couplings X and Y, and 
%various masses and lifetimes.

\begin{comment}
Define compressed and uncompressed? Mention ISR/FSR? (Adam)
\end{comment}


%\section{Status of experimental searches for dark matter and supersymmetry}
%\section{Status of searches for dark matter, supersymmetry and long-lived 
%particles}
\section{Status of searches for supersymmetry and long-lived particles}

Current DM/SUSY limits (CMS/ATLAS and briefly compare to DD, mention X GeV 
limits on 
mediator/DM/gluino/LSP), very constrained hence look for LL signatures, we 
think we’ve excluded X GeV gluinos but that’s not true if they’re LL.
Mono-X but monojet strongest constraints.

Current LL limits (CMS/ATLAS).
Many possible types of striking signatures -- displaced leptons/photons/jets, 
disappearing/kinked tracks, stopped particles, etc.
Not many with MET, first LL interpretation of 
prompt analysis (as long as decays within tracker can still reconstruct with 
standard algorithms, plus ISR), briefly mention complementarity (sensitive to 
whole range of lifetime, sub-cm and all mass splittings).

see llp thesis and workshop
%https://indico.cern.ch/event/671803/

LL: see paper (refs 36-46).


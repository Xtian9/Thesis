\chapter{Results and interpretation}
\label{chap:results}

All signal model studies and systematics go first.

First LL interpretation of prompt analysis. say why this is 
interesting/important (benchmark for future LL searches, see where sensitivity 
to LL currently lies - surprisingly already quite sensitive)


\section{Systematic uncertainties on signal model simulation}
1-2 pages.

Similar to backgrounds.
Luminosity, trigger, MC stat, pileup, b-tagging, JEC, ISR, genmet.

Additional systs for LL.
b-tagging, odd jet (chf, jet id), trigger.

Summarise in table or words typical size of systs for relevant models.


\section{Statistical model}
2-3 pages.

Model the bins and number of events and uncertainties and correlations between 
bins as a likelihood function (probability of observing the observed data given 
some model parameters). This will then be fitted/maximised/used to make the 
background estimations and perform statistical tests to determine limits.

Maybe describe how minimisation is done (gradient descent bla migrad bla).


\section{Results under background-only hypothesis}
3-5 pages.

CR fit ("predictions") and full fit (background-only).

Mountain range plots (2 pages).

Distributions of pulls.

Pulls on nuisances.

No excess observed hence set limits.

\section{title}
2-3 pages.
Hypothesis testing, setting limits, asymptotic CLs
See Nick.

\section{Limits}
Exclusion 




%%%%%%%%%%%%%%%%%%%%%

Acceptance times efficiency.
Most sensitive bins.
Some LL studies - in results section?


DMLL: my presentations, raffaele, oliver exo workshop (future prospects).
Take home message: (see presentation - lack of coverage for models with 
compressed (N2,N1) and small ctau). Prompt search retains some sensitivity 
for long (short) gluino (DM) lifetimes (and is the most sensitive sub-cm and 
lifetimes beyond the detector, hence continue with prompt search in future), 
although clearly could be improved with a dedicated search/tagger as shown by 
the b-tag effects. Also compressed DM can be much improved by removing dphi cut.



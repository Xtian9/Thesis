\chapter{Conclusion}
\label{chap:conclusion}
The extremely successful standard model of particle physics was completed with 
the discovery of the Higgs boson at the LHC in 2012. However, it is clear that 
the standard model is not a complete theory. Its shortcomings 
include the lack of a dark matter particle and the non-inclusion of the 
gravitational force, %and a Higgs mass fine-tuning issue, 
among others. There are many theories beyond the 
standard model, such as supersymmetry, that attempt to solve its problems. Some 
of these models, such as split supersymmetry, predict the existence of massive 
long-lived particles. The LHC presents an excellent opportunity to search for 
new physics.

This thesis has presented an inclusive search for supersymmetry with the CMS 
detector in events containing missing transverse momentum and at least one jet. 
The data correspond to an integrated luminosity of 35.9~\ifb of proton-proton 
collisions at a centre-of-mass energy of 13~TeV at the LHC. The analysis has 
been optimised for new physics signatures from promptly decaying particles, but 
is nevertheless sensitive to signatures from long-lived particles, providing 
complementary coverage to dedicated searches at the LHC, in particular for 
lifetimes of $c\tau \lesssim 1$~cm. This is the first 
interpretation of a `prompt' new physics search in the context of long-lived 
particles.

Candidate signal events are categorised according to the number of jets, the 
number of b-tagged jets, the total hadronic transverse energy and the missing 
hadronic transverse energy. The large QCD background is suppressed with 
dedicated kinematic variables. The dominant standard model backgrounds, namely 
\znnj, \wlj and \ttbar are estimated using single and double muon control 
regions, and appropriate uncertainties are assigned. No significant excess 
above the standard model expectation in the observed number of events is found. 
Statistical hypothesis tests are performed to set upper limits at 95\% 
confidence level on the cross sections of simplified models of supersymmetry. 

% in summary, squarks over 1 TeV, gluinos almost 2 TeV, LSP 0.5 TeV (squarks) 
%or 1 TeV (gluinos)
For models with promptly decaying SUSY particles, gluinos are excluded up to 
almost 2~TeV, and squarks are excluded up to over 1~TeV. 
For the split SUSY model, gluino masses up to 1750 and 900~GeV are 
excluded for gluinos with $c\tau=1$~mm and metastable gluinos, respectively. 
These results do not consider the model-dependent interactions of R-hadrons 
with the detector material. The sensitivity of this search is only moderately 
dependent on these matter interactions for models with $c\tau \gtrsim 1$~m, 
while no dependence is found for models with \ctau below 1~m.

There has been no evidence for physics beyond the standard model so far. 
However, the LHC still presents an excellent opportunity to discover new 
physics within the next decade, as it will continue to operate until a total 
integrated luminosity of 3000~\ifb. As well as exploiting the larger dataset, 
more sophisticated search techniques can be developed. For example, a displaced 
jet tagging algorithm based on deep learning methods, similar to the b-tagging 
algorithm, would significantly improve the acceptance for long-lived particles 
over a wide range of lifetimes.
% HL-LHC: dozen years after 2025, lumi = 5x10^34
% mu2e (COMET)
% DM (LZ)
% there is still motivation for unnatural SUSY 
%https://arxiv.org/pdf/1407.4081.pdf
%Next steps – use ML to exploit whole range of decay lengths. OB: The 
%traditional MET searches are therefore typically far from being optimal for 
%long-lived scenarios.
% need to train for each lifetime
% other improvements: use calo jets rather than PF so don't get killed by chf

\begin{comment}
Prompt search retains some sensitivity 
for long (short) gluino (DM) lifetimes (and is the most sensitive sub-cm and 
lifetimes beyond the detector, hence continue with prompt search in future), 
although clearly could be improved with a dedicated search/tagger as shown by 
the b-tag effects.

Mention improvements using ML tagger for all displacements (extending the 
b-tagging algorithm) and list of things to take advantage in OB DM LL page 7.
see Rob_EXO_LL_tagger.
exploit jet id variables (eg chf etc), use ML/DL

see thomas strebler/mattkomm
displaced jet tagger
https://arxiv.org/pdf/1711.09120.pdf
\end{comment}



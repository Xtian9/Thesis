%% Title
\titlepage[Imperial College London\\Department of Physics]{%
  A thesis submitted to Imperial College London\\ for the degree of Doctor 
  of Philosophy}

%% Abstract
\begin{abstract}%[\smaller \thetitle\\ \vspace*{1cm} \smaller {\theauthor}]
  %\thispagestyle{empty}
  %\LHCb is a \bphysics detector experiment which will take data at the 
  %\unit{14}{\TeV} \LHC accelerator at \CERN from 2007 onward\dots
  %The title-page should be followed by an abstract consisting of no more
  %than 300 words.
  An inclusive search for supersymmetry in final states with jets and missing 
  transverse momentum is performed in proton-proton collisions at a 
  centre-of-mass energy of 13~TeV. A data sample corresponding to an integrated 
  luminosity of 35.9~\ifb recorded by the CMS detector in 2016 at the CERN LHC 
  is analysed. The observed signal candidate event counts are found to be in 
  agreement with the standard model expectation. The result is interpreted in 
  the context of simplified models of split supersymmetry. Lower limits are 
  placed on the mass of a long-lived gluino for a wide range of lifetimes. 
  %ranging from a promptly decaying to a metastable gluino
  Gluino masses up to 1650, 1750 and 900~GeV are excluded at 95\% confidence 
  level for promptly decaying gluinos, gluinos with a lifetime of 1~mm, and 
  metastable gluinos, respectively. These results provide complementary 
  coverage to dedicated searches for long-lived particles at the LHC.
\end{abstract}


%% Declaration
\begin{declaration}
  %This dissertation is the result of my own work, except where explicit
  %reference is made to the work of others, and has not been submitted
  %for another qualification to this or any other university. This
  %dissertation does not exceed the word limit for the respective Degree
  %Committee. 
  %Declaration of Originality
  %You must include a short statement in your own words, that the work is
  %your own and that all else is appropriately referenced
  I declare that this thesis is the result of my own work. All work produced by 
  others is referenced appropriately. The analysis strategy described in 
  Chap.~\ref{chap:analysis} was developed collaboratively within a team of 
  researchers, of which I was a key member and large contributor. The 
  interpretation in long-lived supersymmetry described in 
  Chap.~\ref{chap:results} is the result of my own independent work.
  \vspace*{1cm}
  \begin{flushright}
    Christian Laner
  \end{flushright}
  \vspace*{1.5cm}
	{\textit{The copyright of this thesis rests with the author and is made
	available under a Creative Commons Attribution Non-Commercial No
	Derivatives licence. Researchers are free to copy, distribute or
	transmit the thesis on the condition that they attribute it, that
	they do	not use it for commercial purposes and that they do not alter,
	transform or build upon it. For any reuse or redistribution,
	researchers must make clear to others the licence terms of this
	work}}
\end{declaration}


%% Acknowledgements
\begin{acknowledgements}
  %Of the many people who deserve thanks, some are particularly prominent,
  %such as my supervisor\dots
  Thank you to everyone I have worked with and who has supported me during this 
  journey. Oliver Buchmueller for his supervision and for providing 
  me with this opportunity. Rob Bainbridge, Stefano Casasso, Sarah Malik and 
  Bjoern Penning for their continual support and sharing of knowledge. The 
  entire RA1 team for their hard work and extraordinary teamwork, in particular 
  Mark Baber, Shane Breeze, Matt Citron, Adam Elwood, Ben Krikler and Lucien 
  Lo. Fellow members of CERN and the Imperial HEP group for providing an 
  enriching environment. And last but not least, thank you to my family, 
  particularly my parents and my wife, for their loving and relentless support.
\end{acknowledgements}


%% Preface
%\begin{preface}
%  This thesis describes my research on various aspects of the \LHCb
%  particle physics program, centred around the \LHCb detector and \LHC
%  accelerator at \CERN in Geneva.
%
%  \noindent
%  For this example, I'll just mention \ChapterRef{chap:SomeStuff}
%  and \ChapterRef{chap:MoreStuff}.
%\end{preface}

%% ToC
\tableofcontents


%% Strictly optional!
%\frontquote{%
%  Writing in English is the most ingenious torture\\
%  ever devised for sins committed in previous lives.}%
%  {James Joyce}

%% I don't want a page number on the following blank page either.
\thispagestyle{empty}

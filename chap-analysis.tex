\chapter{Search strategy}
\label{chap:analysis}

This chapter describes the analysis strategy of a search for physics 
beyond the standard model in proton-proton collisions at a centre of mass 
energy of 13~TeV. The search is performed in final states containing missing 
transverse momentum and at least one jet. 

The search is designed to have sensitivity to a wide range of new physics 
models that involve the production of a weakly interacting 
particle (WIMP), such as dark matter or the lightest supersymmetric particle. 
The search has been optimised for signatures in which the WIMP is produced from 
prompt decays at the primary collision vertex. However, as will be discussed in 
Chap.~\ref{chap:results}, the search is also sensitive to signatures in which 
the WIMP is produced at a displaced vertex following the decay of a long-lived 
particle. %This will be the focus of the interpretation.
% OB: prompt MET signatures

%explain hard scatter somewhere?
% overall momentum
%is an imbalance in transverse momentum in the final state, as they travel 
%through the detector without interacting %undetected
% this whole paragraph might need to go somewhere else (maybe intro)
In the proton-proton collisions, the net momentum of the colliding partons in 
the plane transverse to the beam direction is effectively zero, whereas the 
longitudinal momentum is not necessarily so. In order to conserve momentum, the 
outgoing particles produced in the collision must therefore have an overall 
transverse momentum of zero. As WIMPs do not interact with the detector 
material, the measured net transverse momentum in the event will be non-zero. 
This non-zero ``missing transverse momentum'' is the key signature of such 
particles. In addition, at a hadron collider such as the LHC, the dominant 
production is via the strong interaction, and hence jets are readily produced 
either in association with the WIMP, or as initial or final state radiation 
(ISR, FSR). For these two reasons the search is performed in final states 
containing jets and missing energy. At least one jet is required in order for 
the missing momentum to be defined and for the event to be triggered.

A missing energy signature is not unique to WIMPs, however, and is also present 
in certain standard model processes. Neutrinos (produced in the decays of Z and 
W bosons, for example) are also weakly interacting and undetectable at CMS. It 
is also possible for particles to be over or under-measured, thereby 
introducing a ``fake'' momentum imbalance. This type of background arising from 
energy mismeasurements is suppressed as much as possible (to lt1\% of the total 
background) using the variables described in Sec.~X. The remaining standard 
model background (that involving neutrinos) must be estimated as precisely as 
possible, using a combination of theory calculations, simulation, and 
calibrations in data. This is described in Sec.~X. One can then look for a 
statistically significant excess in the data above the expected amount of 
standard model background that would be an indication of the observation of 
physics beyond the standard model. The statistical analysis is covered in 
Chap.~\ref{chap:results}.

inclusiveness
The ways in which a WIMP/BSM signature can manifest are numerous
SR
CRs
binning in four variables

Similar searches for supersymmetry have been performed in Runs 1 and 2 of the 
LHC, at centre of mass energies of 8 and 13~TeV, and for a range of 
integrated luminosities. These can be found in Refs.~[1,2,3,4,5]. These 
searches are used as a basis for the analysis described in this thesis. A 
series of developments and optimisations have been made in order to adapt the 
analysis for the higher centre of mass energy and larger amount of data 
collected. In addition, the interpretations in dark matter and long-lived 
particles described in Chap.~X are a novelty to this search.

%has ptmiss been defined?

This chapter is organised as follows: Section bla describes bla etc.

\begin{comment}
Variable definitions, trigger, selections, binning, signal and control regions, 
background estimations, systematic uncertainties, likelihood model

See AN and paper (old ones too) and theses

This search does not employ specialized reconstruction techniques [36–41] that 
target long-lived gluinos



Inclusive, jets + MET search for new physics
(lots of params in susy)
►Low thresholds of HT > 200 GeV, MHT > 130 GeV, Njet >= 1
►Maximise sensitivity by binning in Njet, Nb, HT, MHT ►Using dedicated 
variables αT, Δφ* to strongly suppress the QCD background ►Data-driven 
estimation of EWK and QCD backgrounds using several control regions

Intro/overview (1-2 pages) to analysis (typical first slide of a presentation) 
- jets plus MET final states, inclusive to wide range of SUSY/DM models, low 
thresholds, binning in 4 variables to maximise sensitivity, QCD suppression, 
data-driven estimations.
Need to mention SR and CRs (muon and dimuon).
Basically need to give 1-2 page overview such that it's clear roughly what our 
cuts are, the dominant backgrounds, the main cuts, the SR and CRs.
Refer to previous analysis results (7, 8 TeV, 2.6, 12.9 fb).

\end{comment}

%\section{Overview of analysis strategy}

%\subsection{title}
Dataset used is 36.9 pm x fb-1 and corresponds to the p-p run of 2016.

\section{Physics objects}
2-3 pages

Using reconstruction algorithms described in Detector Chapter.
Additional requirements are imposed in the analysis.
Jets and energy sums form key component in the search.
Electrons and photons are vetoed in the SR.
Muons are vetoed in the SR and selected in the CRs.
Isolated tracks are vetoed in both SR and CRs.
See AN section 6.

\subsection*{Jets}

antikt 0.4.
pt 40 eta 2.4
CHS.
loose ID (chf, nhf, etc).
b-tagging medium working point CSVv2IVF.

\subsection*{Photons}

Vetoed in SR.
Tight working point. Isolation. Pileup correction. Some other requirements.
pt 25 eta 2.5

\subsection*{Electrons}

Vetoed in SR.
Loose working point. Isolation. Pileup correction. Some other requirements.
pt 10 eta 2.5

\subsection*{Muons}

Vetoed in SR, selected in CRs.
Loose, tight.
Bunch of requirements.
mini-isolation.

\subsection*{Isolated tracks}

Vetoed in SR and CRs.
pt 10, dz, isolation.

\subsection*{Energy sums}

HT and MHT computed using jets as described above.
MET computed using all PF candidates in the event, with type 1 corrections. 
Used in MT and MHT/MET cuts.

\section{Baseline selections}
1 page

Summary table of selections at some point.

Events collected by triggers as described later in Section X.

Wants jets and missing energy. Veto photons and leptons and SITs. Jet pt and 
eta requirements. MET filters (bullet list). Beam halo CHF cuts. Primary vertex 
selection. HT and MHT. Forward jet veto here or QCD rejection?

Beam halo plots - jet phi data/MC with and without cut (and jet CHF?).


\section{Standard model backgrounds}
2-3 pages

After these requirements, still left with significant SM backgrounds from 
electroweak and QCD processes.

\subsection{Electroweak processes}

Dominant are Z and W.
	
Z is irreducible. Neutrinos.
	
W/ttbar is when lepton lost - not reconstructed or out of acceptance.
	
Plus minor/residual backgrounds - single top, diboson, Higgs, etc.
	
\subsection{QCD processes}

Different QCD mechanisms (bullet list plus my studies) - detector effects, fake 
MET, mismeasurement, below threshold, heavy flavour.

\section{QCD background rejection}
2-3 pages.

Various ways of killing it: alphat, bdphi, mht/met.
Propaganda plots.

\subsection{alphaT}

\subsection{bdphi}

\subsection{missing energy ratio MHT/MET}

\section{Event selection}
2-3 pages

Now define the cuts to suppress the QCD, the definition of the signal region 
where we expect the signal, and the definition of the control regions that are 
used to estimate the SM backgrounds.

Big summary table of all selections/regions.

\subsection{Signal region}

alphat (summarise HT-dependent cuts in table), bdphi, mht/met cuts.
Anything else?

\subsection{Control regions}

Enriched in background they want to estimate. No overlap with SR events. 
Ignored lepton in sums to mimic/proxy the SR. Exactly the same cuts as SR 
except for inversion of muon veto to selection, plus some other differences 
described in the following.

Background estimation described in Sec X (EWK) and Y (QCD).

\subsubsection{mujets}

Exactly one muon that passes requirements mentioned before. DeltaR. MT. No 
alphat or bdphi.

\subsubsection{mumujets}

Exactly two muons opposite charge. DeltaR. Mll.

% \subsubsection{single photon}
% used in validation of nb, mHT?

\subsubsection{QCD sidebands} %hadronic control regions

bdphi and MHT/MET sidebands. Enriched in QCD.

\section{Event categorisation}
2 pages.

See AN.

Signal region binning in njet, nb, ht, mht. 
Jet pt (mono sym asym).

Same bins in single mu as SR.
Slightly different in double mu - only two nb bins. Explain why? Higher stats. 
Show nb extrapolation validation (AN) or just summarise briefly in couple of 
sentences (similar to MHT validation).

Table of bins.

%\subsection{Simplified binning scheme}
% Probably not necessary, just show results with nominal bins

%\subsection{Nb extrapolation}

\section{Triggers}
5 pages.

See Mark.
List of SR and CR triggers.
Efficiencies.

\section{Simulation samples}
0.5 pages.
Maybe put here the "datasets" section of data and MC samples (see Matt).

\section{Corrections to simulation}
3-5 pages

Simulation modelling is not perfect. Need to correct it using scale factors by 
comparing data and MC. These corrections introduce systematic uncertainties in 
the background estimation, that will be described later in Sec X.

\subsection{Pileup}
\subsection{b-tagging efficiency}
Reweighting formula.
\subsection{Trigger efficiency}
\subsection{Lepton and photon reconstruction, identification, isolation and 
triggering efficiency}
Tag and probe.
\subsection{Top quark pT}
\subsection{Cross-sections}
Sideband corrections.
Summarise in table.

\subsection{Plus more?}
NLO, pdf/scale
Signal (maybe later): nISR, gen met

\section{b-tag formula method}
1-2 pages.

See Burton. AN Sec 4.4.
Relevant but be brief.
How much improves limits? Just refer to T1bbbb as T1qqqqLL 1 mm is very similar.

Purpose: reduce stat uncertainty of simulation in higher nb bins (where signal 
can lie).

Method: show the formula and explain it.

Summarise in table/plot or one-two sentences how much the stat unc is reduced. 
Maybe also how much the limits are improved (Lucien did this).

Formula systematics.
%\subsection{Systematics}

\section{Estimation of electroweak background processes}
Predict normalisation using CRs (data-driven to reduce reliance on MC). Use MHT 
templates from MC.

\subsection{nj, nb, ht dimensions}
1 page

Transfer factor method.

\subsection{MHT dimension}
2 pages

Take templates because don't want to bin CRs too finely (lose statistical power 
- curse of dimensionality).

Validation: Check data MC ratio is flat in CRs. Assign syst as described later 
in Sec X.

\section{Estimation of QCD background processes}
2 pages

Method.

Validation.

Plot of estimated yields per bin.

\section{Systematic uncertainties on background estimation}

These are explained below and also summarised, with representative magnitudes, 
in the big summary table of systs.

\subsection{MC-based}
3-4 pages (maybe 2 pages just of plots).

Known theoretical and experimental uncertainties.
Largely cancel out in the TF ratio.

Refer to Section of corrections to simulation. These are the associated 
uncertainties.

Pileup, JEC, b-tagging, lepton, photon, trigger, top pt, W/tt, NLO, ttbar nISR.

Example 2D plots of variations in the bins.

\subsection{Closure tests}
2-3 pages.

Probe additional sources of systematics.

Define method.

Go through each test (there's not many now) and describe what it's probing: 
extrapolation in alphat and bdphi, W polarisation, SITV.

Closure plots.

\subsection{MHT templates}
2 pages. See Matt and old ANs.

Derivation of uncertainties. Vs njet and ht.

Plot illustrating size of uncertainty in each bin.
